\documentclass[12pt,a4paper]{scrartcl}
\usepackage[utf8]{inputenc}
\usepackage[german]{babel}
\usepackage{graphicx}

\begin{document}

\section{Zerstörungsfreie Materialprüfung Mittels Ultraschallmesstechnik}
\subsection{Abgrenzung von anderen Testverfahren}

Obwohl die zerstörungsfreie Prüfung mittels Ultraschall in der heutigen Wirtschaftslage unverzichtbar ist, kann mithilfe der Ultraschallmesstechnik nicht jeder Materialtest durchgeführt werden, da auch diese durch die Grundlagen der Physik eingeschränkt wird.
%da diese durch die physikalischen Gegebenheiten eingeschränkt sind. \\
%Somit werden weitere Materialeigenschaften verwendet, um technische Teile auf ihre Funktionalität zu prüfen. 
Um diese Grenzen zu überwinden, wird auf andere Materialeigenschaften zurückgegriffen.
Dabei handelt es sich um optische, thermische, elektrische, magnetische und atomare Eigenschaften des Prüflings. 
Jede der Methoden bietet dabei eigene Vor- und Nachteile und ist damit für spezielle Prüfvorgänge geeignet. \\
%Weiterhin unterscheiden sich die Verfahren auch entscheidend in ihre Komplexität und Voraussetzung and Technik und Fachwissen. 
Grundsätzlich unterscheiden sich diese Verfahren in ihren Anforderungen an technische Geräten und dem benötigten Fachwissen, diese zu bedienen und den Prüfvorgang ordnungsgemäß durchzuführen. 
Diese versucht man natürlich, möglichst gering zu halten, um viele Bauteile effizient prüfen zu können.
%besseres Wort Bauteile?
Daher sind die am meisten verwendeten Verfahren die Sichtprüfung und die Farbeindringprüfung. 
Bei diesen wird entweder mit bloßem Auge oder mithilfe von farbigen Flüssigkeiten die Oberfläche von Prüflingen auf Fehler begutachtet. 
%Entscheidend ist dabei, den Vorgang schnell, einfach und ohne technischen Aufwand durchführen zu können. 
Dies kann schnell, einfach und ohne technischen Aufwand durchgeführt werden.
Doch durch den Ablauf beim Prüfen werden auch die Nachteile dieser Verfahren klar, denn unter Verwendung des Auges oder von Kameras kann nur die Oberfläche von Prüfkörpern beurteilt werden. \\
Dies ist für einige Teile ausreichend, doch besonders stark beanspruchte oder sicherheitsrelevante Teile dürfen auf keinen Fall versagen. 
Oft reicht dabei schon ein kleiner Riss im Inneren, um genau dieses Versagen herbeizuführen. 
%Daher ist es hier besonders wichzig, auch das Innere des Prüfkörpers auf Fehler zu untersuchen. 
Daraus ergibt sich die Notwendigkeit, in diesen Fällen den Prüfkörper auch auf innere Beschädigungen zu untersuchen.
%Hierfür stehen auch Methoden zur Verfügen. 
Um dies durchzuführen stehen ebenfalls mehrere Methoden zur Verfügung.
Abhängig von Größe und Material des Prüflings sind manche Testverfahren besser geeignet als andere. 
So basieren z.B. manche Verfahren auf der elektrischen Leitfähigkeit des Prüfkörpers und sind daher auf bestimmte Materialien beschränkt.
Andere hingehen erfordern geschlossene Sicherheitsräume, wie z.B. die Röntgenprüfung, und können daher schlecht zur Prüfung von bereits verbauten Teilen verwendet werden.\\
Allerdings kann auch durch die verwendete Methode selbst eine Einschränkung erfolgen. So ist z.B. beim Prüfen mit Ultraschall die Materialdicke abhängig vom verwendeten Prüfgerät auf 5-8m beschränkt, da sonst Ungenauigkeiten entstehen könnten.\\
% 5-8m ausschreiben?
Das Ultraschallverfahren, dessen Simulation Thema dieser Arbeit ist, bietet insgesamt eine gute Kombination praxisrelevanter
% praktikabler? praktischer? sonstiges?
 Eigenschaften. So können bei relativ geringem Technischen Aufwand und unter Verwendung mobiler Prüfgeräte kleine bis mittelgroße Prüfkörper relativ genau auf innere Fehler untersucht werden.

% zu viel dies(e) ???

\end{document}